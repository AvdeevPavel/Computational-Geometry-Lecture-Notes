\section{Определение полигона}

Пусть $\Gamma \subset \plain$ --- замкнутая ломаная без cамопересечений. $\Gamma = \{v_0 v_1 v_2 ... v_{n-1} v_n \},
~v_i = v_{n + i}$.
\begin{figure}[h]
\includegraphics{gamma.eps}
\end{figure}
Определим функцию $isects(p, ray) \to \{0, 1\}$, $p \in \plain \setminus \Gamma$ возвращающую четность числа пересечений 
луча $ray$ с началом в точке $p$ с $\Gamma$. При подсчете числа пересечений необходимо учитывать два особых случая: 
\begin{itemize}
    \item $ray \cup \{v_{i-1}v_{i}v_{i+1}\} = v_i$, 
    \item $ray \cup \{v_{i-1}v_{i}...v_{j}v_{j+1}\} = \{v_{i}...v_{j}\}$. 
\end{itemize}

\begin{figure}[h]
\begin{minipage}[h]{0.49\linewidth}
\center{\includegraphics[width=0.9\linewidth]{isect-1-a.eps} \\ a}
\end{minipage}
\hfill
\begin{minipage}[h]{0.49\linewidth}
\center{\includegraphics[width=0.9\linewidth]{isect-1-b.eps} \\ b}
\end{minipage}
\caption{особый случай пересечения первого типа}
\label{pic:isect_1}
\end{figure}

В случае (\ref{pic:isect_1}) будем считать, что если вершины $v_{i-1}$, $v_{i+1}$ лежат по одну сторону от луча $ray$~(a), 
то пересечение в точке $v_i$ не засчитывается, а если по разные (b), то засчитывается ровно одно.

\begin{figure}[h]
\begin{minipage}[h]{0.49\linewidth}
\center{\includegraphics[width=0.9\linewidth]{isect-2-a.eps} \\ a}
\end{minipage}
\hfill
\begin{minipage}[h]{0.49\linewidth}
\center{\includegraphics[width=0.9\linewidth]{isect-2-b.eps} \\ b}
\end{minipage}
\caption{особый случай пересечения второго типа}
\label{pic:isect_2}
\end{figure}

Случай (\ref{pic:isect_2}) на самом деле не сильно отличается от случая (\ref{pic:isect_1}), если представить себе, что 
цепь $\{v_iv_{i+1}...v_j\}$ схлопывается в одну точку. Соответственно, если вершины $v_{i-1}$, $v_{j+1}$ лежат по одну сторону 
от луча $ray$~(a), то пересечение по цепи $\{v_iv_{i+1}...v_j\}$ не засчитывается, а если по разные (b), 
то засчитывается ровно одно.

\begin{statement}
Функция $isect(p, ray)$ зависит только от $p$. 
\begin{proof}
Функция $isect(p, ray)$ непрерывна по аргументу $ray$ и принимает дискретные значения, значит она постоянна.
\end{proof}
\end{statement}

\begin{definition} Задавшись замкнутой ломаной $\Gamma$, определим полигон $P$ следующим образом. \\
$$int(P) \eqdef \{ p \; | \; isect(p) = 1 \},$$
$$ext(P) \eqdef \{ p \; | \; isect(p) = 0 \},$$
$$P \eqdef int(P) \cup \Gamma.$$
\end{definition}

Интуитивно понятные термины границы, внутренности, связности полигона совпадают с тем, как они определяются в топологии.

\begin{statement}
$\Gamma$ --- граница $P$.
\end{statement}
\begin{statement}
$int(P)$ --- внутренность $P$.
\end{statement}
\begin{statement}
$int(P)$, $ext(P)$ --- связны. $int(P) \cup ext(P) = \plain \setminus \Gamma$ --- несвязно.
\end{statement}

Зачем нужно столь формализованное опеределение полигона? Одно из применений --- алгоритм определения принадлежности точки 
полигону, для которого обоснование корректности тривиально. Соответсвует ли полигон, определенный таким образом, 
тем свойствам, которые ожидаются? Давайте покажем парочку.

\begin{statement}
Выпуклый полигон (в смысле пересечение конечного числа полуплоскостей) --- полигон в определенном выше смысле.
\end{statement}
\begin{statement}
Если вершины границы полигона конечны, то полигон ограничен.    
\end{statement}
