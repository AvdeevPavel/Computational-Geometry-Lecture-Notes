\section{Определение полигона}

Пусть $\Gamma \subset \plain$ --- замкнутая ломаная без cамопересечений. $\Gamma = e_0 \cup e_1 \cup ... \cup e_{n-1}, 
e_i = [v_i, v_{i+1}], v_i = v_{n + i}$.
\begin{figure}[h]
\includegraphics{gamma.eps}
\end{figure}
Определим функцию $isects(p, ray) \to \{0, 1\}$, $p \in \plain \setminus \Gamma$ возвращающую четность числа пересечений 
луча $ray$ с началом в точке $p$ с $\Gamma$. При подсчете числа пересечений необходимо учитывать два особых случая: 
\begin{itemize}
    \item $ray \cup \{v_{i-1}v_{i}v_{i+1}\} = v_i$, 
    \item $ray \cup \{v_{i-1}v_{i}...v_{j}v_{j+1}\} = e_{i} \cup ... \cup e_{j-1}$. 
\end{itemize}

\begin{figure}[h]
\begin{minipage}[h]{0.49\linewidth}
\center{\includegraphics[width=0.9\linewidth]{isect-1-a.eps} \\ a}
\end{minipage}
\hfill
\begin{minipage}[h]{0.49\linewidth}
\center{\includegraphics[width=0.9\linewidth]{isect-1-b.eps} \\ b}
\end{minipage}
\caption{особый случай пересечения первого типа}
\label{pic:isect_1}
\end{figure}

В случае (\ref{pic:isect_1}) будем считать, что если вершины $v_{i-1}$, $v_{i+1}$ лежат по одну сторону от луча $ray$~(a), 
то пересечение в точке $v_i$ не засчитывается, а если по разные (b), то засчитывается ровно одно.

\begin{figure}[h]
\begin{minipage}[h]{0.49\linewidth}
\center{\includegraphics[width=0.9\linewidth]{isect-2-a.eps} \\ a}
\end{minipage}
\hfill
\begin{minipage}[h]{0.49\linewidth}
\center{\includegraphics[width=0.9\linewidth]{isect-2-b.eps} \\ b}
\end{minipage}
\caption{особый случай пересечения второго типа}
\label{pic:isect_2}
\end{figure}

Случай (\ref{pic:isect_2}) на самом деле не сильно отличается от случая (\ref{pic:isect_1}), если представить себе, что 
цепь $\{v_iv_{i+1}...v_j\}$ схлопывается в одну точку. Соответственно, если вершины $v_{i-1}$, $v_{j+1}$ лежат по одну сторону 
от луча $ray$~(a), то пересечение по цепи $\{v_iv_{i+1}...v_j\}$ не засчитывается, а если по разные (b), 
то засчитывается ровно одно.

\begin{statement}
Функция $isect(p, ray)$ зависит только от $p$. 
\begin{proof}
Функция $isect(p, ray)$ непрерывна по аргументу $ray$ и принимает дискретные значения, значит она постоянна.
\end{proof}
\end{statement}

\begin{definition} Задавшись замкнутой ломаной $\Gamma$, определим полигон $P$ следующим образом. \\
$$int(P) \eqdef \{ p \; | \; isect(p) = 1 \},$$
$$ext(P) \eqdef \{ p \; | \; isect(p) = 0 \},$$
$$P \eqdef int(P) \cup \Gamma.$$
\end{definition}

Интуитивно понятные термины границы, внутренности, связности полигона совпадают с тем, как они определяются в топологии.

\begin{statement}
\label{stat:Bounds}
$\Gamma$ --- граница $P$.
\begin{proof} 
    Напомним, что граница это множество точек, в каждой окрестности которых есть точки как принадлежащие $P$, так и нет.
    \begin{description}
        \item [Точка $p$ лежащая строго внутри $e_i$ --- граничная.] 
            Рассмотрим окрестность $p$ с радиусом $r = \min_{j \neq i}{dist(e_j, p)}$, внутри которой выберем две точки 
            $p_1, p_2 \notin e_i$  так, чтобы $p_1$, $p$, $p_2$ лежали на одной прямой, причем точка $p$ находилась между 
            $p_1$ и $p_2$. Ясно, что луч пущенный из $p_1$ в сторону $p_2$ имеет на одно пересечение с $\Gamma$ больше, 
            чем его подлуч с началом в $p_2$, а значит, либо $p_1 \in P, p_2 \notin P$, либо наоборот.
        \item [Точка $p$ совпадающая с вершиной $v_i$ --- граничная.] Этот случай отличается от предыдущего тем, что
            $r = \min_{j \neq i - 1, i}{dist(e_j, p)}$ и $p_1, p_2 \notin e_{i-1}, e_i$.
        \item [Точкa $p$ не принадлежащая $\Gamma$ не принадлежит границе.] Рассмотрим окрестность $p$ с радиусом 
            $r = \min dist(p, \Gamma) = \min_{i=1..n}{dist(p, e_i)}$ (из последнего равенства видно, что $r > 0$). 
            Луч пущенный из всех точек этой окрестности, так чтобы он проходил через $p$ пересекается с $\Gamma$ такое же число 
            раз, как и луч пущенный из $p$, а значит, все точки этой окрестности принадлежат $P$, если $p \in P$, 
            и не принадлежат иначе.
    \end{description}
\end{proof}
\end{statement}

\begin{statement}
\label{stat:Int}
$int(P)$ --- внутренность $P$.
\begin{proof} 
    Напомним, что внутренность это максимальное по включению открытое подмножество.
    \begin{description}
        \item [$int(P)$ --- открыто.] Следует из последнего пункта доказательства предыдущего утверждения.
        \item [$int(P)$ --- максимальное из открытых.] 
            Так как точки $\Gamma$ --- граничные, они не могут входить во внутренность.
    \end{description}
\end{proof}
\end{statement}

\begin{statement}
$int(P)$, $ext(P)$ --- связны. $int(P) \cup ext(P) = \plain \setminus \Gamma$ --- несвязно.
\begin{proof}
    Напомним, что необходимым и достаточным условием связности множества в $\mathbb{R}^2$, является то, что любые две 
    точки множества можно соединить непрерывнм путем, принадлежащим множеству.
\begin{description}
    \item [Существует путь между произвольными точками $p_1, p_2 \in int(P) \left( ext(P) \right)$.]
    \item [Не существует пути между $p_1 \in int(P)$, $p_2 \in ext(P)$.] 
\end{description}
\end{proof}
\end{statement}


Зачем нужно столь формализованное опеределение полигона? Одно из применений --- алгоритм определения принадлежности точки 
полигону, для которого обоснование корректности тривиально. Соответсвует ли полигон, определенный таким образом, 
тем свойствам, которые ожидаются? Давайте покажем парочку.

\begin{statement}
Выпуклый полигон (в смысле пересечение конечного числа полуплоскостей) --- полигон в определенном выше смысле.
\begin{proof}
Как и все почти, доказывается по индукции.
\end{proof}
\end{statement}

\begin{statement}
Если вершины границы полигона конечны, то полигон ограничен.    
\end{statement}
