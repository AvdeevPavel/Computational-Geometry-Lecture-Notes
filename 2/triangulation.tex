\section{Триангуляция полигона}

\textbf{Art Gallery Problem (\href{http://en.wikipedia.org/wiki/Art_gallery_problem}{точная формулировка в Wikipedia}).} 
Дана картинная галерея в форме полигона, необходимо расставить в ней минимальное возможное число охранников так, 
чтобы каждая точка галереи была под наблюдением.
\begin{proof}[Идея решения]
Понятно, что для любого выпуклого полигона (в частности, треугольника) хватит одного охранника, 
расположенного в любой точке, к примеру, в вершине. Поэтому, мы бы решили задачу, если бы нам удалось разбить исходный полигон 
на треугольники, и расставить охранников в некоторые из вершин получившихся треугольников так, чтобы каждому треугольнику была
инцидентна хотя бы одна вершина с охранником. Скажем, если бы вершины треугольников были покрашены в три цвета
так, чтобы вершины каждого трегольника были разных цветов, то ответом могли бы стать все вершины одного из цветов. 
Если множество вершин треугольника совпадает с множеством вершин полигона (путь его мощность $n$), то по принципу Дирихле 
мощность ответа не превосходит $\lfloor n / 3 \rfloor$. 

С другой стороны существует примеры полигонов со сколько угодно большим числом вершин $n$, для которых $\lfloor n / 3 \rfloor$ ---
нижняя граница ответа.

Для формализации разбиения на полигона на треугольники и алгоритма покраски вершин введем понятие \itshape{трианугляции полигона}.
\end{proof}

\begin{definition}
    Диагональ полигона --- отрезок, концы которого --- вершины полигона, а внутрении точки принадлежат внутренности полигона. 
\end{definition}
Будем говорить, что две диагонали не пересекаются, если не пересекаются множества их внутренних точек.
\begin{definition}
    Триангуляция полигона --- максимальное по включению множество попарно непересекающихся диагоналей.
\end{definition}
\begin{statement}
    \label{stat:DiagonalExistance}
    Если число вершин полигона больше трех, он имеет диагональ.
\end{statement}
\begin{statement}
    \label{stat:PolygonSubdivision}
    Диагональ разбивает полигон. А именно, путь полигон $P$ имеет границу $\Gamma = \{v_0 v_1 v_2 ... v_{n-1} v_n \}$, 
    и $v_i v_j$ его диагональ, обозначим как $P_1$ полигон с границей $\Gamma_1 = \{v_i v_{i+1} ... v_j v_i \}$, 
    как $P_2$ полигон с границей $\Gamma_2 = \{v_j v_{j+1} ... v_i v_j \}$. Тогда $P_1 \cup P_2 = P$, $P_1 \cap P_2 = v_i v_j$.
\end{statement}
\begin{statement}
    То что мы определили как ''триангуляция'' названо так неслучайно, а именно, триангуляция разбивает полигон на треугольники.
\end{statement}
Последнее утверждение кажется очевидным, и действительно, с помощью утверждений (\ref{stat:DiagonalExistance}) и 
(\ref{stat:PolygonSubdivision}) его несложно доказать индукцией по числу вершин: база --- треугольник, 
переход осуществляется разбиением полигона на два с помощью какой-нибудь диагонали.
Но оно нетривиально и, не определяя полигон формально, доказать его без рукомахательства непросто. Чтобы продемонстрировать это,
попробуйте доказать что любой политоп триангулируем, то есть разбивается на тетраэдры (\textbf{подсказка: } это неправда).
