\section{Применение выпуклых оболочек}
\subsection[title]{Поиск наиболее удаленных точек} 

\begin{statement}
Наиболее удаленные точки лежат на границе выпуклой оболочки.
\begin{proof}
Пусть это не так. Тогда, исходя из определения выпуклой оболочки, точки лежат в выпуклом полигоне. Проведем прямую через эти две точки. 
Очевидно, что прямая пересечет границу полигона в двух местах, так как фактически мы пустили из двух точек по лучу. Получается, что расстояние между точками, в которых пересеклась прямая с границей, больше, чем между данными по условию точками. Получили противоречие.        
\end{proof}
\end{statement}

\begin{definition} 
Опорная прямая - это прямая, проходящая через точку выпуклой оболочки и обладающая тем свойством, что выпуклая оболочка лежит по одну сторону от нее.  
\end{definition}

\begin{statement}
Через две наиболее удаленные точки можно провести две параллельные опорные прямые. 
(to do здесь нужен рисунок. У меня почему-то не вставляет eps файлы. Гуглил не помогло. Если додумаюсь то нарисую сразу же) 
\begin{proof} 
Рассмотрим, для одной из наиболее удаленной точек все опорные прямые. \\
Для каждой такой прямой рассмотрим параллельную ей прямую, прохожящую через другую удаленную точку\\
В итоге мы найдем параллельную опорную прямую и во второй точке.  
\end{proof}
\end{statement}

Исходя из утверждений, доказанных выше, предъявим метод поиска наиболее удаленных точек. \\
\subsubsection{Алгоритм вращающихся крепер} 
(to do нужен рисунок)
\begin{enumerate}
	\item Выберем на выпуклой оболочки две точки A и B и построим две парралельные опорные прямые.  
	\item Будем вращать эти прямые по часовой стрелке с одинаковой угловой скоростью.
	\item Когда прямая начнет пересекать целый отрезок, возьмем его крайную точку.
	\item Из этой крайней точки строим опорную прямую и переходим на шаг 2.
	\item Так перебираем все пары, пока не получим пару (B, A).    
\end{enumerate}
(to do а расстояние мы ищем между точками на шаге 4, так ведь?)

\begin{statement}
Работа данного алгоритма - O(M$\log{n}$), где M - это количество точек на выпуклой оболочке, что такое n - ? я не вспомнил. 
\begin{proof}
Исходя из алгоритма, мы пройдем все точки выпуклой оболочки $\longrightarrow$ O(M).
Скорее всего $\log{n}$завязано на вращении 
\end{proof}
\end{statement}

\subsection[title]{Поиск наибольшего диаметра}
Тут мне многое не понятно. 

\begin{statement}
Сложность $\omega(n\log{n})$
\begin{proof}
Сведем доказательство к поиску элементов из двух множеств. 
\end{proof}
\end{statement}

И еще тут окружность диаметром p = 2. Там мы ищем диаметр, но не понятно для чего это. Поясните кратко, пожалуйста. 
