\section[title]{Триангуляция Делоне}

\subsection[title]{Цель триангуляции}
(нужен рисунок, какой-либо триангуляции) \\
Дано множество точек $v = \{v\}_{i=1}^{n}$ и функция $f:V\rightarrow \Re$. Хотим интерполировать эту функцию.  

Напомним, что такое интерполяция.
 
\begin{definition} 
Интерполяция - это способ нахождения промежуточных значений величины по имеющемуся дискретному набору известных значений.
\end{definition}

Для решения этой проблемы разобьем $\omega$ на треугольники, так что бы $\omega$ стала триангулированной выпуклой оболчкой. 
Дальше сопоставляем вершинам значения функции $f$ в точке, а если не попадает будем брать центроид треугольника (что не попадает? похоже на ересь не хватает каких-то слов).
Этим занимается метод конечных элементов.

(Можете еще раз рассказать те вводные слова)

\subsection[title]{Определение триангуляции}
\begin{definition} 
Триангуляция множества точек - это максимальный планарный граф.
\end{definition}

\begin{definition} 
Максимальный планарный граф - это такой граф, в который нельзя добавить ни одного ребра, не нарушив планарность. 
\end{definition}

\begin{statement}
Все грани триангуляции множества точек, кроме внешней, являются треугольниками.
\begin{proof} 
Пусть у нас в триангуляции нашелся не треугольник, тогда по доказанному ранее утверждению(в теме про триангулцию полигона) существует диагональ. 
Проведя ее, получаем либо треугольник, либо еще пару не треугольников и ищем диагональ для них. В итоге все грани, кроме внешней - треугольники. 
\end{proof}
\end{statement}

Ясно, что триангуляций на заданном множестве может быть несколько. Как же тогда выяснить, какая триангуляция лучше? \\
Для этого надо ввести критерии, по которым будем сравнивать триангуляции. Скорее всего, первое что приходит на ум, это идея: 
{\bf "Чем меньше треугольников в триангуляции, тем лучше".} Чтобы показать, что это не правда, докажем следующее утверждение.

\begin{statement}
Количество треугольников в триангуляции на заданном множестве точек фиксировано. 
\begin{proof} 
Для доказательства нам потребуется формула Эйлера для графов:
$$
F + P - 2 = E 
$$
где $|V| = n$ - множество точек, F - грани из трех ребер, a E - количество ребер, которые можем посчитать. \\
А так же такой факт для графов, что 
$$
\sum_{i=1}^{n}v_i = 2E
$$
Где $v_i$ - степень вершины i, а E - как и ранее, количество ребер.\\
Посчитаем сумму степеней вершин: \\
to do еще подумаю как доказать, тупо надо формулу написать и посчитать. на паре же мы это смогли как-то сделать
Кстати формула выведенная на паре$(f-1)3 + h = 2E$ - не работает для четырехугольника $3*3 + 4 != 2*8$, нужна такая $3f+h = 2E$
\end{proof}
\end{statement}

Предъявим критерии, которые будут давать ответ на вопрос, какая из триангуляций лучше.

\begin{enumerate}
	\item $E_{d} = \frac{1}{2}\int\limits_{\Omega}|\nabla f|^2d\Omega$ - энергия Дирихле. \\
		Необходимо минимизировать энергию Дирихле. $E_{d}\rightarrow min$ \\
		Плохой случай: (to do рисунок) - линейная интерполяция по отрезку. Надо с этим разобраться.
	\item Максимизация минимального угла треугольника.
	\item $\sum_{t \in T}^{} \frac{a^2+b^2+c^2}{S}$, где $T$ - множество треугольников из триангуляции.
		Мы должны минимизировать этот ряд. 
\end{enumerate}

\subsection[title]{Триангуляция Делоне}
\subsubsection[title]{Определения}
\begin{definition} 
Триангуляция Делоне - никакие точки из триангуляции не содержатся в окружности, описанной возле любого треугольнкиа из триангуляции.
\end{definition}

to do рисунок для двух смежных треугольников, когда удолетворяет триангуляции и когда нет.

\begin{statement}
Если критерий Делоне выполняется для каждой пары смежных треугольников, то он выполняется для любого треугольника из триангуляции. 
\end{statement}

\subsubsection[title]{Доказательство существования}
Для доказательства существования, такой триангуляции предъявим конструктивный алгоритм. 
\begin{algorithm}
Построение любой триангуляции.
\begin{enumerate}
	\item Сортируем координаты по x. 
	\item Берем две точки в порядке возрастания x. Строим прямую.
	\item Берем точку, если она лежит на прямой, повторяем этот шаг.
	\item Если точка не лежит на прямой, соединяем её с точками, полученными ранее на прямой. Эта точка становится "началом" 
		прямой, переходим к шагу 3. 
\end{enumerate}
Ведь мы выходим из алгоритма когда обработали каждую точку? 
\end{algorithm}

\begin{definition} 
Операция Flip - это смена общей стороны у двух смежных треугольников (поворот диагонали в четырехугольнике). 
\end{definition}
to do нужен рисунок иллюстрирующий это. 

\begin{statement}
Если для пары смежных треугольников не был выполнен критерий Делоне, то после Flip-а он будет выполняться.
\begin{proof} 
Это надо доказывать же? (Если да, то тут вроде необходима школьная геомтерия )
\end{proof}
\end{statement}

После определения операции Flip алгоритм дальнейших действий очевиден. Нужно для каждой пары смежных треугольников, у которых не выполняется критерий Делоне, произвести операцию Flip. Чтобы понять, что данные действия конечны, докажем следующее утверждение.  

\begin{statement}
Нужно сделать конечное число Flip.
\begin{proof} 
надо читать, то что кинули. С минимизацией объема и то что $(x, y) \rightarrow (x, y, x^2 + y^2)$
\end{proof}
\end{statement}

\subsubsection[title]{Доказательство выполнимости критериев}

