\section{Определение выпуклой облочки}

\begin{definition}
\itshape{Выпуклая оболочка} множества точек $S$ ($CH(S)$) это минимальное по включению выпуклое множество, 
содержащее все точки из $S$.
\end{definition}

\begin{statement}
Если $|S| < \infty$, то $CH(S)$ есть выпуклый полигон, причем вершины $CH(S)$ принадлежат $S$.
\end{statement}
\begin{proof}
Достаточно доказать следующие два утверждения.
\begin{itemize}
\item
  \label{FstItem}
  Если найдется выпуклый полигон с вершинами из $S$, содержащий все точки из $S$, он будет выпуклой оболочкой $S$
  (доказывается сначала для треугольника, а потом для произвольного выпуклого полигона, триангуляцией).
\item Найдется выпуклый полигон, содержащий все точки из $S$ (доказывается индукцией по мощности $S$).
\end{itemize}
\end{proof}

\begin{statement}
\label{MostLeft}
Самая левая точка $S$ принадлежит $CH(S)$.
\end{statement}

\begin{statement}
Выпуклая оболочка может быть найдена за время $\Theta(n \log h)$, где $n = |S|$, $h = |CH(S)|$.
\end{statement}
\begin{proof}
Оценка $O(n \log h)$ доказывается сведением к сортировке, $\Omega(n \log h)$ доказывается конструктивно (алгоритм Чена).
\end{proof}
